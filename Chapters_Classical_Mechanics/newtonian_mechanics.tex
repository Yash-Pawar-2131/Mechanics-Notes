\section{Introduction}
	\subsection{}

\section{Momentum}
In \textbf{Newtonian mechanics}, \textbf{linear momentum}, \textbf{translational momentum}, or simply \textbf{momentum} is the product of the \textbf{mass} and \textbf{velocity} of an object. It is a \textbf{vector quantity}, possessing a magnitude and a direction. If $m$ is an object's mass and $\mathbf{v}$ is its velocity, then the object's momentum $\mathbf{p}$ is
\[
\mathbf{p} = m\mathbf{v}.
\]

Momentum depends on the frame of reference, but in any inertial frame it is a conserved quantity, meaning that if a closed system is not affected by external forces, its total linear momentum does not change.

It is an expression of one of the fundamental symmetries of space and time: translational symmetry.

In continuous systems such as electromagnetic fields, fluid dynamics and deformable bodies, a momentum density can be defined, and a continuum version of the conservation of momentum leads to equations such as the Navier–Stokes equations for fluids or the Cauchy momentum equation for deformable solids or fluids.

	\subsection{Dependence on reference frame}
	
	\subsection{conservation}

	\subsection{Angular Momentum}

\section{Dynamics}
	\subsection{Newton's Laws of Motion}
		\subsubsection{Newton's First Law}
		\subsubsection{Newton's Second Law}
		\subsubsection{Newton's Third Law}
	\subsection{Euler's Laws of Motion}
		\subsubsection{Euler's First Law}
		\subsubsection{Euler's Second Law}	
	\subsection{Roational Dynamics}
		\subsubsection{Torque}
	\subsection{Statics}

\section{Oscillation}
	\subsection{Harmonic Oscillator}