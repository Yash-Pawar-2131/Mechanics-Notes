\section{Coordinate System}
A \textbf{coordinate system} is a system that uses one or more numbers, or \textbf{coordinates}, to uniquely determine the position of the \textbf{points} or other geometrci elements on a \textbf{manifold} (such as \textbf{Euclidean Space}).

\section{Transformation}

\section{Inertial Frame of Reference}
Within the realm of \textbf{Newtonian Mechanics}, an \textbf{inertial frame of reference}, or inertial reference frame, is one in which \textbf{Newton's first law of motion} is valid. However, the \textbf{principle of special relativity} generalizes the notion of inertial frame to include all physical laws, not simply Newton's first law.
\\\\
Newton viewed the first law as valid in any reference frame that is in uniform motion relative to the fixed stars; this is, neither rotating nor accelerating relative to the starts. Today the notion of "absolute space" is abandonded, and an inertial frame in the field of \textbf{classical mechanics} is defined as:
\begin{definition}
An \textbf{inertial frame of reference} is one in which the motion of a particle not subjected to forces is in a straight line at constant speed.
\end{definition}
Hence, with respect to an inertial frame, an object or body \textbf{accelerates} only when a physics force is applied, and (following Newton's first law of motion), in the absence of a net force, a body at \textbf{rest} will remain at rest and a body in motion will continue to move uniformly - that is, in a straigth line and at constant speed. Newtonian inertial frames transform among each other according to the \textbf{Galiliean group of symmetries}.
\\\\
If this rule is interpreted as saying that straight-line motion is an indication of zero net force, the rules does not identify inertial reference frames because straight-line motion can be observed in a vareity of frames. If the rule is interpreted as defining an inertial frame, then we have to be able to determine when zero net force is applied.

\section{Galiliean Transformation}
\subsection{Galiliean Invariance}

\section{Euler Angles}

\section{Generalized Coordinate}

\section{Constraints and Degrees of Freedom}
\subsection{Holonomic Constraint}
\subsection{Non-holonomic Constraint}