\section{Lorentz Transformation}
Define an event to have spacetime coordinates $(t, x, y, z)$ in system $S$ and $( t', x', y', z')$ in a reference frame moving at a velocity $\mathbf{v}$ with respect to that frame, $S'$. Then the Lorentz transformation specifies that these coordinates are related in the following way:
\[
\begin{aligned}
t' & = \gamma \ (t - vx/c^2) \\
x' & = \gamma \ (x - v t) \\
y' & = y \\
z' & = z ,
\end{aligned}
\]
where
\[
\gamma = \frac{1}{\sqrt{1 - \frac{v^2}{c^2}}}
\]
is the \textbf{Lorentz factor} and $c$ is the \textbf{speed of light} in vacuum, and the velocity $\mathbf{v}$ of $S'$, relative to $S$, is parallel to the $x$-axis. For simplicity, the $y$ and $z$ coordinates are unaffected; only the $x$ and $t$ coordinates are transformed. These Lorentz transformations form a one-parameter group of linear mappings, that parameter being called $rapidity$.
\\\\
Solving the four transformation equations above for the unprimed coordinates yields the inverse Lorentz transformation:
\[
\begin{aligned}
t & = \gamma (t'+vx'/c^{2}) \\
x & = \gamma (x'+vt') \\
y & = y' \\
z & = z'.
\end{aligned}
\]